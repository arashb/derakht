\documentclass[border=10pt]{article}
%%%<
\usepackage[english]{babel}
\usepackage{graphicx}
\usepackage{tikz}

% \usepackage{verbatim}

% \usepackage{float}
% \usepackage{graphicx}
\usepackage{caption}
\usepackage{subcaption}
% \usepackage{wrapfig}
%%%>

\begin{document}
\pagenumbering{gobble}
\begin{figure}
  \centering
  \begin{subfigure}[b]{0.9\textwidth}
      \input{../c}
      \input{../v}
    \caption{\textbf{Original Approach:} The trees are partitioned independently.}
  \end{subfigure} ~ %add desired spacing between images, e. g. ~, \quad, \qquad, \hfill etc. %(or a blank line to force the subfigure onto a new line) 

  \begin{subfigure}[b]{0.9\textwidth}
      \input{../cm}
      \input{../cm}
    \caption{\textbf{Complete Merge:} The trees are merged and then partitioned.}
  \end{subfigure} ~ %add desired spacing between images, e. g. ~, \quad, \qquad, \hfill etc. %(or a blank line to force the subfigure onto a new line) 

  \begin{subfigure}[b]{0.9\textwidth}
      \input{../csm}
      \input{../vsm}
      \caption{\textbf{Semi-Merge:} The trees are partitioned based on the merged tree without actual merge. When necessary, the breaking point is created in the tree.}
  \end{subfigure} ~ %add desired spacing between images, e. g. ~, \quad, \qquad, \hfill etc. %(or a blank line to force the subfigure onto a new line) 
  % \caption{Illustration of different merging schemes for concentration and velocity trees.}
  \label{fig:oa}
\end{figure}

\end{document}
